%---------------------------------------------------------------------------%
%->> Frontmatter
%---------------------------------------------------------------------------%
%-
%-> 生成封面
%-
\maketitle% 生成中文封面
\MAKETITLE% 生成英文封面
%-
%-> 作者声明
%-
\makedeclaration% 生成声明页
%-
%-> 中文摘要
%-
\intobmk\chapter*{摘\quad 要}% 显示在书签但不显示在目录
\setcounter{page}{1}% 开始页码
\pagenumbering{Roman}% 页码符号

% 摘要
% AllDifferent约束是一种最基本的全局约束,已被用于建模许多经典的约束满足问题(CSPs)。为了解决这个约束,以前的研究主要集中在提出过滤算法来减少变量的值域,从而修剪回溯搜索。然而,由于AllDifferent约束的内在复杂性,对于现有的基于搜索的求解器来说,解决包含大规模AllDifferent约束的CSPs仍然具有挑战性。
% 考虑到AllDifferent约束的特性,我们在本文中提出了一种高效的局部搜索算法,名为AllDiff-LS。我们首先将一组AllDifferent约束表示为图,并通过两个多项式时间的化简规则简化图。然后,我们提出了一种新的两步策略来选择移动,以及精心设计的禁忌和重启策略,使算法能更有效地探索搜索空间。
% 在一组代表性的基准测试上的实验结果显示,AllDiff-LS找到的解决方案比最先进的完整和启发式方法更多,特别是在大规模实例上,运行时间大大减少。

% 介绍CP,介绍AllDifferent约束
约束满足问题是人工智能和运筹学中的一个数学问题定义。在约束满足问题中,需要找出满足一组约束的解,其中约束是对可能解的限制条件。约束满足问题可以用来描述许多实际问题,例如时间表编制、资源分配、电路设计等。约束规划是一种解决约束满足问题的编程范式,其核心思想是将问题建模为一组变量和约束,用高级、声明式的方式描述问题,然后使用约束传播、回溯搜索等算法寻找满足所有约束的解。其中,AllDifferent约束是一种常见且基本的全局约束,它要求一组变量的取值都不相同,从而有效地减少搜索空间,排除许多显然不可能的解,使搜索过程更加高效。这种约束在许多经典的约束满足问题中都有应用,例如数独问题,八皇后问题等。针对此类约束,目前已经提出了许多过滤算法和搜索算法,以剪枝和加速求解。然而由于AllDifferent约束的内在复杂性,对于现有的基于搜索的求解器来说,解决包含大规模AllDifferent约束的约束满足问题仍然具有挑战性。

% 介绍求解方法
本文深入探讨AllDifferent约束的特性,提出了用于求解AllDifferent约束的高效的局部搜索算法。首先,本文将问题中的一组AllDifferent约束转化为图表示,该图将AllDifferent约束分解为多个子约束,从而更细粒度地描述了AllDifferent约束涉及的变量和变量表达式,以及变量表达式之间的关系。接下来,鉴于图中涉及的两类关系,本文设计了两个多项式时间的化简规则来简化这个图,对问题进行预处理来减小问题的规模。之后,本文通过局部搜索算法对问题进行求解。由于修改候选解的操作涉及到对变量和赋值的选择,本文提出了一种分步选择操作的策略来修改候选解,对于选择的两个步骤分别设计新的打分函数,并通过图中的邻域关系在平局出现时打破平局。除此之外,本文还精心设计了禁忌和重启策略。禁忌策略可以防止算法陷入局部最优,而重启策略则可以在搜索陷入困境时及时调整搜索的方向,这两种策略的设计都使算法能更有效地探索搜索空间。在禁忌策略中,本文将操作的禁忌分为了三部分,从而适配分步选择操作的策略。而在重启策略中,本文维护了候选解池,并在获得次优解时对约束图进行加权操作,使算法能更有效地探索搜索空间。

% 介绍实现的工具
利用上述算法,本文实现了一个名为AllDiff-LS的高效约束求解工具,专门用于处理AllDifferent约束。该工具支持各种算术表达式,并能处理包含其他二元约束的约束满足问题。通过将问题编码为约束,可以利用AllDiff-LS来求解这些约束,并提供满足所有约束的解。本文在多种经典的CSP问题上进行了实验,如数独、N皇后、全间隔和正交拉丁方问题。实验结果显示,与针对性的启发式求解算法以及基于启发式和完备的通用求解器相比,本文的方法在求解效率上表现更优,特别是在大规模实例上,该算法能够解决更多的挑战性问题。

\keywords{约束规划,AllDifferent约束,局部搜索}% 中文关键词
%-
%-> 英文摘要
%-
\intobmk\chapter*{Abstract}% 显示在书签但不显示在目录

The concept of constraint satisfaction problems constitutes a mathematical formulation within artificial intelligence and operations research. In the realm of constraint satisfaction problems, the objective is to find solutions that satisfy a set of constraints, which impose limitations on the feasible solutions. Constraint satisfaction problems can be employed to describe a wide range of practical issues, such as timetable creation, resource allocation, circuit design, and more. Constraint programming serves as a programming paradigm for tackling constraint satisfaction problems. It revolves around modeling problems using variables and constraints, expressed in a sophisticated and declarative manner. Subsequently, algorithms like constraint propagation and backtracking search are utilized to explore solutions that fulfill all the imposed constraints. Among these techniques, the AllDifferent constraint emerges as a common and fundamental global constraint. It mandates that a group of variables must assume distinct values, effectively reducing the search space and eliminating many clearly infeasible solutions. As a result, the search process becomes more efficient. This particular constraint finds practical applications in various classical constraint satisfaction problems, including sudoku and the eight queens problem. To address such constraints, numerous filtering and search algorithms have been developed to enhance pruning and expedite the problem-solving process. Nonetheless, due to the intrinsic complexity of the AllDifferent constraint, solving constraint satisfaction problems that involve large-scale AllDifferent constraints remains a challenging task for existing search-based solvers.

In this paper, a thorough investigation of the characteristics of the AllDifferent constraint is conducted, and an efficient local search algorithm for solving the AllDifferent constraint is proposed. Firstly, a set of AllDifferent constraints in the problem is transformed into a graph representation. This graph decomposition breaks down the AllDifferent constraints into multiple sub-constraints, providing a more granular description of the variables and variable expressions involved in the AllDifferent constraint, as well as the relationships between variable expressions. Next, considering two types of relationships present in the graph, two polynomial-time reduction rules are devised to simplify the graph and preprocess the problem, thereby reducing its size. Subsequently, a local search algorithm is employed to solve the problem.

Regarding the modifications made to candidate solutions which involve selecting variables and assignments, a stepwise selection strategy is proposed to modify the candidate solutions. For each of the two steps in the selection process, new scoring functions are designed, and ties are broken using the neighborhood relationships in the graph when a tie occurs. Furthermore, carefully crafted taboo and restart strategies are implemented. The taboo strategy prevents the algorithm from getting trapped in local optima, while the restart strategy adjusts the search direction when the search gets stuck. Both strategies enable the algorithm to explore the search space more effectively. In the taboo strategy, the taboo set for operations is divided into three parts to accommodate the stepwise selection strategy. In the restart strategy, a candidate solution pool is maintained, and when a suboptimal solution is obtained, the constraint graph undergoes weighted operations, enabling the algorithm to explore the search space more effectively.

Using the aforementioned algorithm, I have implemented an efficient constraint-solving tool called AllDiff-LS, specifically designed for handling AllDifferent constraints. This tool supports various arithmetic expressions and can handle constraint satisfaction problems that include other binary constraints. By encoding the problem as constraints, I can utilize AllDiff-LS to solve these constraints and provide solutions that satisfy all the imposed constraints. I conducted experiments on several classical CSP problems, such as Sudoku, N-Queens, All-Interval Series, and Orthogonal Latin Squares. The experimental results demonstrate that our approach outperforms targeted heuristic solvers as well as heuristic and complete general-purpose solvers in terms of solving efficiency, particularly for large instances. This algorithm proves to be capable of solving more challenging problems in such situations.

\KEYWORDS{Constraint Programming, AllDifferent Constraints, Local Search}% 英文关键词
%---------------------------------------------------------------------------%
