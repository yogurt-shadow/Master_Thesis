\chapter{LS\_NRA的工具实现与实验结果}\label{chap:implementation}

本章节主要介绍了LS\_NRA的整体框架和算法细节。除了前面两章介绍的胞腔跳跃缓存机制和等式松弛机制,我们还为非线性问题操作的停滞设计了简单的前瞻机制。然后,我们详细讨论本工具的具体实现,包括预处理阶段、重启策略、线性方程的快速运算、变量可行域的计算以及参数设置。

\section{启发式移动和前瞻机制}
如前文所述,非线性算数理论的一个挑战就是并不总是存在单变量的关键移动来满足一个特定的约束。之前的做法更多依赖于沿着固定直线的参数方程替代\cite{LiXZ23},更一般的做法需要用到柱形代数分解或多项式优化等理论。特别地,在SMT-LIB测试样例中,一类来自于生物网络\cite{AkutsuHT08}的名为Sturm-MBO的样例覆盖了大量的非常复杂的多项式,并且规定所有变量只能取正值。当多项式包含很多变量时,目前的启发式搜索方法很难找到满足约束的赋值。

本小节提出一种新的应对这种问题的方法,并且保证每次仍然只移动一个变量。为方便说明,我们称一个文字\textbf{停滞}当该文字目前处于未满足状态,并且不存在任何的单变量关键移动使其满足。给定一个目前处于停滞状态的文字$l$,我们首先在$l$中选择一个系数(根据其他变量的赋值决定)不为0的变量$x$,然后启发式地选择一系列候选移动作为$x$接下来的赋值。对于每一个候选值来说,我们计算文字$l$在赋值后是否仍然处于停滞状态。我们优先选择那些没有处于停滞状态的候选赋值。

假定当前赋值$x_0$,变量$x$的可行域为$I$,启发式的移动选择包括以下几种:
\begin{enumerate}
    \item 可行域$I$的每个区间中,靠近边界值的有理数和整数。有理数被设定为与边界值相差$10^{-4}$。比如,对于可行域$[11.2, 15.1]$,选择的移动有$\{11.2, 12, 15, 15.1\}$。
    \item 比$x_0$大于或小于的临近整数。比如,对于当前赋值$x_0 = 13.5$,选择的移动有$\{13, 14\}$。
    \item 从区间$[\frac{x_0}{2}, x_0)$中均匀地选取三个值,从区间$(x_0, 2x_0]$中均匀选取三个值。
\end{enumerate}

第一类操作反映了变量$x$约束的信息。第二类操作借鉴了随机游走的思想,并且优先选择简单的整数值。第三种操作是最常见的,允许在更小或者更大的分数上进行搜索。算法\ref{alg:look-ahead}总结了以上的基本思想。我们用集合$S$收集候选赋值的移动,然后循环验证集合$S$中的每一个元素。如果文字$l$在赋值之后没有陷入停滞状态,那么返回这个赋值。否则,返回集合$S$中的随机元素。

\begin{algorithm}[t]
    % \small
    \caption{Heuristic choice of candidate values and look-ahead for critical moves}
    \label{alg:look-ahead}
    \textbf{输入}: Literal $l$ that is stuck\\
    \textbf{输出}: Candidate variable $v$ and new value $x_1$
    
    \begin{algorithmic}[1] %[1] enables line numbers
        \Statex \hrulefill
        \STATE x \leftarrow variable in polynomial  $l$ with non-zero coefficient;
        \STATE S \leftarrow heuristic move selection for variable $x$;
        \FOR {value $x_1$ in $S$}
            \IF {$l$ has critical move after assigning $x$ to $x_1$}
                \RETURN $x, x1$;
            \ENDIF
        \ENDFOR
        \STATE $x_1 leftarrow$ random chosen value in $S$;
        \RETURN $x, x_1$
    \end{algorithmic}
\end{algorithm}

\section{实现细节}
LS\_NRA主要在Z3定理证明器上实现,并且使用Z3原生的多项式和代数数库。我们的工具借鉴了Z3中MCSAT算法的实现,共享了文字、子句的数据结构,但是内部算法逻辑和代码实现完全独立。下面我们将主要介绍LS\_NRA的几个模块。

\textbf{预处理阶段:} 预处理阶段主要负责化简主要的子句形式,以及简单的变量替换,为后续的主要搜索过程提供方便。
\begin{itemize}
    \item \textbf{子句化简:} 将同时出现的子句$p \le 0$和$p \ge 0$合并为$p = 0$。
    \item \textbf{变量替换:} 在形如$c \dot x + q = 0$的等式约束中,其中$c$是常数,$q$是次数最多为1的多项式并且最多包含2个变量,替换掉变量$x$。这里我们限制$q$的形式以降低变量替换的复杂度。
\end{itemize}

\textbf{重启策略:} 我们设计了一个双层的重启策略,参数分别为$T_1$和$T_2$,值为100。其中一次\textbf{小重启}在$T_1$次迭代没有改进后执行,随机选择未满足子句中的一个变量修改赋值。在$T_2$次小重启之后,一次\textbf{大重启}会重置所有变量的赋值。

\textbf{线性方程的快速运算:} 本文的实根隔离是基于Z3现有的多项式操作库完成的,其原理是牛顿迭代。但是当变量在多项式中以线性项出现时,我们可以通过斜率快速计算可行域,而非使用更通用的实根隔离函数。

\textbf{参数设置:} 本工作的可调参数设置如表\ref{tab:parameter}所示。

\begin{table*}[]
    \centering
    \resizebox{0.7\linewidth}{!}{
        \begin{tabular}{c | c | c}
            \\\hline
            符号 & 参数说明 & 预设值 \\\hline
            $sp$ & PAWS加权策略的概率 & 0.006\\\hline
            $T_1$ & 小重启执行所需的无提升迭代次数 & 100\\\hline
            $T_2$ & 大重启执行所需的小重启次数 & 100\\\hline
            $\epsilon_v$ & 等式松弛所需的复杂度阈值 & $10^{-4}$\\\hline
            $\epsilon_c$ & 等式松弛的程度 & $10^{-4}$\\\hline
        \end{tabular}
        }
        \bicaption{算法的可调参数设置。} {Tunable parameters of the algorithm.}
\label{tab:parameter}
\end{table*}
