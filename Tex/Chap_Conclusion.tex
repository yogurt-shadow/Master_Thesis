\chapter{总结与展望}\label{chap:method}

本章节主要对前面介绍的工作进行系统性的总结概括,归纳其中的主要贡献点,并对下一步的工作进行展望。

\section{工作总结}

包含AllDifferent约束的约束满足问题既重要又具有挑战性。虽然已有学者提出了一些算法来解决这些问题,但它们在大规模问题实例上的扩展性不佳。
本工作主要提出了一种新的用于求解AllDifferent约束的局部搜索算法——AllDiff-LS,具体地,它由三部分组成:将CSP转化为局部搜索算法易处理的图结构,构造适用于给定图结构的局部搜索算法,以及设计用于该算法的重启策略。在三、四章中,本文对这三部分进行了详细的介绍。通过一系列实验,表明本文的算法在解决AllDifferent约束中的有效性,它可以在几分钟内解决大规模和复杂的问题实例。

对于CSP中的AllDifferent约束,本文构造了一个异构图称为AllDifferent约束图(ACG),它的思路是使用图结构对这些AllDifferent约束二元分解后得到的二元约束的并集进行表示。由于本文隐去了值域这个信息,一个ACG得以描述整组AllDifferent约束,从而可以施展更强的弧一致性算法。之后,本文定义了图上的赋值操作,以及基于它的化简规则,其核心思想是,根据变量和变量表达式顶点的度对图中的顶点和边进行删减。在对约束化简之后,本文通过随机赋值的方式获得最初的候选解。

本文将初始化后的ACG和候选解作为局部搜索的输入,通过定义状态、移动、评价准则等要素,设计了局部搜索算法。鉴于一个移动可以被分为两部分,本文设计了两步选择的移动策略作为直接选择策略的轮换策略。在此基础上,本文设计了一整套禁忌策略,它由三个小策略组成,前两个策略用于保证算法可以跳出循环,后一个策略则提供了选择操作的快速轮换机制,保证了算法的性能。同时,本文设计了打破对称的策略,借助变量表达式的邻域关系打破平局。

此外,本文设计了基于解池技术的一整套重启策略。首先本文介绍了解池的维护机制,并提出了解的准入和生成策略,这两个策略分别用到了解近似的概念,和对解的扰动策略。
此外,本文介绍了通过解池对ACG进行加权,从而减少对已探索区域的重复探索。最后,本文介绍了动态迭代的思想,在解加入解池和从解池中选择时,通过一定的策略增加下一轮迭代的最大迭代次数,这样做的动机是对复杂区域进行充分的探索,以寻找潜在的更优解。

基于上述思想,本文得到了求解AllDifferent约束的AllDiff-LS算法,通过将等式约束、偏序不等式约束引入到ACG中,本文可以处理比AllDifferent约束更广泛的CSP类型。比较实验和消融实验表明AllDiff-LS算法拥有更好的求解能力,并且证明了算法中涉及的各个策略的有效性。

\section{下一步的工作}

在未来,希望将局部搜索算法更加完善,让其适用于更广泛的约束满足问题上,实现一个更加通用的求解器,用来提升CSP求解的能力和效率。
此外,将启发式策略和完备策略结合在一起对CSP进行求解,是一个有希望的改进方向。
完备求解器一般依赖于传播和搜索,并因此具有较强的推理能力。一方面,在局部搜索之前或中途都可以使用约束传播对解空间进行化简;另一方面,完备搜索算法中关于变量序、分支等的启发式,同样可以应用于局部搜索中,做为指引。
而其中一个思路是先在一个小的搜索窗口中使用完备的求解算法,如果无法求解则转为调用启发式求解方法。

\clearpage