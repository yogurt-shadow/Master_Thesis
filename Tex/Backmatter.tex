%---------------------------------------------------------------------------%
%->> Backmatter
%---------------------------------------------------------------------------%
\chapter[致谢]{致\quad 谢}\chaptermark{致\quad 谢}% syntax: \chapter[目录]{标题}\chaptermark{页眉}
%\thispagestyle{noheaderstyle}% 如果需要移除当前页的页眉
%\pagestyle{noheaderstyle}% 如果需要移除整章的页眉

几年的硕士生涯走到了最后,值此论文完成之际,谨向求学路上所有给予我指导、支持与陪伴的师长、亲友致以最诚挚的感谢。

首先,我要对帮助过我的各位老师表达深深的感谢,包括曾经的导师詹博华副研究员、现任导师张立军研究员。他们作为我的导师帮助我初步拟定了课题的选定、基本的研究路线,并且帮助我养成了研究生该有的成熟心态。除此之外,蔡少伟老师作为约束求解领域的专家也对我的研究给出了十分宝贵的指导意见,吴志林老师也曾在整个毕业流程中多次对我施以援手。软件所的其余导师都无形之中或多或少对我给予了专业上的帮助和指引,他们的言传身教将是我未来学术道路上的重要财富。

其次,我也要向国重实验室的各位兄弟姐妹表达感谢。作为中途换过导师的学生,我曾经流转过多个课题组,他们在很多方面都给予了我极大的帮助。詹博华老师的学生帮助我很好地熟悉了学校的环境和基本的研究思路,蔡少伟老师的学生则更多在算法设计和具体的课题指引上担任起了领路人的帮助,张立军老师的学生则在最后一学年给予了我很多就业和未来发展的关键意见。可以说,国重实验室是一个充满了爱和帮助的大家庭,我很荣幸能够成为其中的一员。

再者,我也要感谢很多形式化社区的学者和从业人员。比如,我的工作更多是基于现有的SMT求解器开发的,因此Z3和cvc5等主要软件的开发团队是我的技术前辈,他们是使我看得更远的巨人,也是我永远要学习的楷模。我在英国参加VMCAI会议时,曾和来自不同国家、不同研究背景的同龄人热烈讨论过,他们对我的意见和对我工作的欣赏是我莫大的欣慰。我也要感谢国内的形式化社区,比如中国计算机学会的形式化方法专委会,他们为我提供了很多关于形式化方法的学习资源和交流机会。当然,我最应该感谢的其实还是投稿VMCAI会议时的无名审稿人,他们的审稿意见让我更好地打磨了自己的作品,让我欣慰地交上了这份答卷。功利地说,如果不是他们最后给出了正面的审稿意见,可能我的论文还未发表,我也未能顺利毕业。

未来,我可能会从事不同的行业和研究方向,但是硕士期间积累的宝贵经历是我人生地巨大财富。


\chapter{作者简历及攻读学位期间发表的学术论文与研究成果}

\section*{作者简历:}

王忠汉,男,辽宁大连人,1999年生,中国科学院软件研究所硕士研究生。

2017年9月——2021年6月,在南开大学电子信息与光学工程学院获得学士学位。

2021年9月——2025年6月,在中国科学院软件研究所攻读硕士学位。

\section*{已发表(或正式接受)的学术论文(加星号的表示共一作者):}

{
\setlist[enumerate]{}% restore default behavior
\begin{enumerate}[nosep]
    \item \textbf{Zhonghan Wang}, Bohua Zhan, Bohan Li, Shaowei Cai. Efficient Local Search for Nonlinear Real Arithmetic. (VMCAI 2024)
\end{enumerate}
}

% \section*{投稿经历}

% {
% \setlist[enumerate]{}% restore default behavior
% \begin{enumerate}[nosep]
%     \item AllDiff-LS: Solving Alldifferent Constraints with Efficient Local Search, AAAI 2023 过第一阶段,未中。
% \end{enumerate}
% }

\section*{参加的研究项目及获奖情况:}

\begin{enumerate}
    \item 参与了课题组可满足性模理论求解工具的工具开发和测试。
    \item 参与了课题组交互式定理证明的讨论。
\end{enumerate}

\cleardoublepage[plain]% 让文档总是结束于偶数页,可根据需要设定页眉页脚样式,如 [noheaderstyle]
%---------------------------------------------------------------------------%
